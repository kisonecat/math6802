\documentclass{homework}
\input{preamble}
\author{Jim Fowler}
\course{Math 6802}
\date{Week 6: Products}

\begin{document}
\maketitle

My original plan was that, on top of the cup product story, we'd take
a brief aside to dig into acyclic models and the Eilenberg-Zilber
theorem.  That story would connect nicely with our having shown that
group homomorphisms $\alpha : G \to G'$ give rise to chain maps
between free resolutions of $G$ and $G'$.  But in discussing this, we
really have plenty on our plate with cup products, so this week will
focus more narrowly on the cup proudct story.

So from Hatcher's \textit{Algebraic Topology}, read the beginning of
\textsection 3.2 on cup products.

The advantage of cup products is being able to regard the cohomology
groups together as a ring.  Previously, we learned that a map $f : X
\to Y$ gives rise to a map $f^\star : H^n(Y) \to H^n(X)$ which is a
homomorphism of groups for each $n$.  But regarding these groups
together, we get a $H^\star(Y) \to H^\star(X)$ which is a homomorphism
of rings, which gives additional restrictions on the possible maps
between spaces.

Another theme, explored heavily in the homework, will be the failure
of commutativity of the cup product on the chain level---and yet the
cup product \textit{is} homotopy commutative, i.e., on the level of
cohomology classes.  Steenrod squares make visible the chain-level
failure of commutativity.

\end{document}
