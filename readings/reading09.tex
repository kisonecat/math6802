\documentclass{homework}
\input{preamble}
\author{Jim Fowler}
\course{Math 6802}
\date{Week 9: Orientations}

\begin{document}
\maketitle

This is the week before Spring Break --- and after the break, we meet
Poincar\'e duality, and given the importance of manifolds, Poincar\'e
duality is an incredibly important topic.  

Working backwards, to make sense of Poincar\'e duality, we need the
idea of an \textbf{orientation.}  You maybe first met such a structure
in a multivariable calculus course (or even a physics course with the
``right-hand rule'').  The goal is a notion of handedness.  For a
vector space, this amounts to putting a basis in order.  There's a
homological shadow of this: for an $n$-manifold $M$, the two
generators of $\mathbb{Z} \cong H_n(M, M \setminus \{ x \})$ can be
viewed as a ``local'' orientation at $x \in M$.  These local
orientations may patch together: for an orientable manifold, there is
a \textbf{fundmental class} $[M] \in H_n(M)$ whose image in each
$H_n(M,M\setminus \{x\})$ is a generator.

Along with the structure of orientations, we'll also riff on cup
products when we introduce the \textbf{slant} and \textbf{cap}
product.  We'll see the cap product and fundamental class again when
we meet Poincar\'e duality. To get started, from Hatcher's
\textit{Algebraic Topology}, begin reading \textsection 3.3 on
orientations and homology; we'll pick up the rest of this after Spring
Break.

\end{document}
