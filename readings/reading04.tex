\documentclass{homework}
\input{preamble}
\author{Jim Fowler}
\course{Math 6802}
\date{Week 4: Functors}

\begin{document}
\maketitle

This week marks our first ``official'' foray into cohomology, but its
more properly about ``functors,'' namely Ext and Tor.  These are
derived functors, related to Hom and $\otimes$, respectively.  What
are derived functors?  Tensoring does not, in general, take exact
sequences to exact sequences, so Tor measures this failure --- of
course, the measured failure \textit{iteslf} need not take exact
sequnces to exact sequences, so for $R$-modules, there are more Tor
functors.

The pedagogical challenge is that some of the students in this course
have seen some homological algebra in the first-year grad courses, but
for other students, this may be the first exposure to derived
functors, and really to homological algebra in general.  This is,
after all, an \textit{algebraic} topology course, so I think it is
appropriate to spend some time focusing on the homological algebra.
The other benefit to spending some lecture time on homological algebra
is to equip you to read Hatcher with more enthusiasm.

So from Hatcher's \textit{Algebraic Topology}, first take a look at
Chapter~3 and even some of \textsection 3.1, which also points the way
to next week's discussion of universal coefficients.  But \textbf{then
focus on \textsection 3.A} which discusses the universal coefficients
theorem for homology, and digs into the ``Tor'' story.  For some of
you, this may be review, but for others, the discussion of Tor could
be quite challenging.

As always, if you have topics you'd like to see us discuss more
significantly during our lecture time, please let me know.  The
success of the course depends on communication.

\end{document}
