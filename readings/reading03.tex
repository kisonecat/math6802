\documentclass{homework}
\input{preamble}
\author{Jim Fowler}
\course{Math 6802}
\date{Week 3: Simplices}

\begin{document}
\maketitle

In this short week, from Hatcher's \textit{Algebraic Topology}, read
\textsection 2.C Simplicial Approximation, which is an invitation to
review simplicial complexes and $\Delta$ complexes.  How far is a
continuous map from being a simplicial map?  What is a
\textbf{simplicial map}?

And why would we want to replace our maps with simplicial maps?  A
continuous map is a rather floppy object, so it is wonderful to find
out that, with a small homotopy, we can regard these floppy objects as
being simplicial, and therefore determined by what they do on the
vertices.  It would have been painful to have had to build our maps
initially in such a rigid fashion, so a theorem that lets us replace
our easily constructed objects by more rigid objects, and therefore
objects which are easier to reason about, is a big win!  We'll reap
our winnings with the \textbf{Lefschetz fixed point theorem} which is
Theorem~2C.3 in Hatcher.  In Hatcher, the conclusion of the theorem is
that $f$ has a fixed point---can you count the fixed points with this
theorem?

And if you make it through \textsection 2.C, then take a look at 3.A
Universal Coefficients for Homology.  We talked a bit about this in
lecture. But here, we'll meet Tor, which will deepen our understand of
homology with coefficients, along with deepening our understanding of
the tensor product.  Throughout this, we are seeing the crucial role
played by chain complexes, so it is helpful to reflect on why chain
complexes are appearing so frequently---what do they help with?  If
you are enjoying digging into this, you might want to review the
homotopy invariance section of Hatcher on Page~110, where you can
review \textbf{chain maps} and \textbf{chain homotopies}.

Why bring this up now?  There's a nice connection between chain maps
and the Lefschetz fixed point theorem.  The \textbf{Lefschetz number}
asks us to compute an Euler characteristic style alternating sum of
traces on the \textit{homology} groups.  What if we did the same
calculation on the level of chains, when the chain groups $C_n$ are
finitely generated abelian groups?  This would be the \textbf{Hopf
trace formula}.

\end{document}
