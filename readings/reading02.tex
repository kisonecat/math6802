\documentclass{homework}
\input{preamble}
\author{Jim Fowler}
\course{Math 6802}
\date{Week 2: Homology and coefficients}

\begin{document}
\maketitle

From Hatcher's \textit{Algebraic Topology}, read \textsection 2.3 The
Formal Viewpoint, which is an opportunity for us to review the
``axiomatic'' perspective on homology.  Also read the first couple
pages of Chapter~3, Cohomology.

With my having been at the JMM last week, we're still in the
introductory period of this course, where we're reviewing homology.
Be sure to use your time wisely to consolidate your prior learning
from Math~6801.  Thinking back, what are you finding confusing about
Math~6801?  Homology of pairs?  Excision?  How comfortable are you
doing these homological calculations?

And why is an axiomatic perspective so important?  It turns out that
there are \textit{other} homology theories, and the Eilenberg-Steenrod
axioms are capturing the essential properties that all homology
theories have in common.  The history is also important: the language
of functors and categories was invented (discovered?) precisely to
make precise the comparison between different homology theories.

\end{document}
