\documentclass{homework}
\input{preamble}
\author{Jim Fowler}
\course{Math 6802}
\date{Week 8: Examples}

\begin{document}
\maketitle

We've done a lot of work this term.  Some of this has been review
(like homology), and some of it has been new content (like K\"unneth).
But what's it all good for?  The machinery can be truly inspiring at
times---recall how we proved K\"unneth by finding a natural
transformation between two cohomology theories, and reflect just how
sophisticated an argument like that is.  But can we actually
\textit{use} this stuff?

This week, with the theme of ``examples,'' is designed to remind us
all that \textit{we can do calculations}.  Of course, all the
homeworks (especially the not always very numerical ``numericals'')
are designed to remind us of this, but let's take time this week to
focus on some special cases and do some calculations.

From Hatcher's \textit{Algebraic Topology}, we'll focus on the part of
\textsection 3.2 called ``spaces with polynomial cohomology.''  This
is also a chance to dig into real and complex and quaternionic
projective space.

\end{document}
