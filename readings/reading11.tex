\documentclass{homework}
\input{preamble}
\author{Jim Fowler}
\course{Math 6802}
\date{Week 11: Duality}

\begin{document}
\maketitle

We're about two-thirds of the way through this course, and as is often
the case at this point, we're reaching a high point: we've worked hard
to dig into cohomology, into products, into orientations, and now we
put all this together in the special case of \textit{manifolds}, and
discover that the cohomology and homology of manifolds are related via
a certain kind of duality.  This fits into the broader theme of
``local to global'' principles throughout mathematics: Poincar\'e
duality is the consequence of piecing together the local data (i.e.,
of being a manifold) to the global situation (i.e., relating the
homology and cohomology of the manifold).

From Hatcher's \textit{Algebraic Topology}, read \textsection 3.3
Poincar\'e Duality.  Superficial evidence for Poincar\'e duality is
that the Betti numbers are palindromic; this was noticed more than 125
years ago.  Our formualtion is more nuanced than this: for an
$n$-dimensional closed orientable manifold, we have
\[
  H^k(M) \cong H_{n-k}(M).
\] This perspective was worked out in the 1930s by \v{C}ech and
Whitney.  The older result for Betti numbers can be recovered by
invoking the Universal Coefficients Theorem.

\end{document}
