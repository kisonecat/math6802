\documentclass{homework}
\input{preamble}
\author{Jim Fowler}
\course{Math 6802}
\date{Week 12: Boundaries}

\usepackage{draftwatermark}
\SetWatermarkText{Draft}
\SetWatermarkScale{5}

\begin{document}
\maketitle

The theme this week is \textbf{boundaries} which is really a theme
throughout the whole course: what is homology? It's cycles modulo
\textit{boundaries}?  When we considered whether there is a retraction
from a compact manifold onto its boundary, we were really identifying
a relationship between ``boundary'' in the sense of homology and
``boundary'' of a manifold---a relationship which the fundamental
class makes clearer.

For this week, from Hatcher's \textit{Algebraic Topology}, continue
reading \textsection 3.3 Poincar\'e Duality.  This provides us with
time to review some of the Poincar\'e duality content from last week,
and also put Poincar\'e duality in a broader framework of other kinds
of duality.  The first is Lefschetz duality, a version of Poincar\;e
duality for manifolds with boundary.  But then we'll also see
Alexander duality, a duality result for the homology of the complement
of a subspace $X$ of a sphere $S^n$.  We'll apply Alexander duality to
study the homology of knot and link complements.

\end{document}
