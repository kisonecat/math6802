\documentclass{homework}
\input{preamble}
\author{Jim Fowler}
\course{Math 6802}
\date{Week 5: Cohomology and coefficients}

\begin{document}
\maketitle

After much waiting, we finally meet cohomology this week.

From Hatcher's \textit{Algebraic Topology}, read \textsection 3.1 on
cohomology groups, and specifically on \textbf{the Universal
Coefficient Theorem}.  But haven't we already seen the Universal
Coefficient Theorem?  We saw \textit{a} universal coefficient theorem,
namely a theorem relating \textit{homology} with integer coefficients
and homology with arbitrary coefficients.  This week, we see a similar
theorem for \textit{co}homology.  The ``Tor'' story from last week
continuse with an ``Ext'' story this week, as expected based on the
homework.

Having to put in twice as much work to produce both a theorem and a
dual theorem, to prove a result about homology and then one about
cohomology \ldots well, you might be thinking all this dualizing is
just a waste of time.  To assuage these feelings, begin reading
\textsection 3.2 on the cup product (but don't read through the
K\"unneth theorem section --- we'll meet that theorem in a couple
weeks).  The cup product provides a ring stucture to cohomology, and
the usefulness of the ring structure will be quickly apparent!  Now,
continuous maps, by functoriality, not only provide maps between
cohomology groups, but provide maps between cohomology \textit{rings},
and by respecting this additional structure, the possible maps are
more restricted.

Next week, we'll take a brief aside away from Hatcher's text to meet
acyclic models and the Eilenberg-Zilber theorem.

\end{document}
