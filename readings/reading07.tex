\documentclass{homework}
\input{preamble}
\author{Jim Fowler}
\course{Math 6802}
\date{Week 7: K\"unneth}

\begin{document}
\maketitle

Last week, we met the cup product --- and there is more work yet to do
to understand the product structure on cohomology.  Until now, we've
been thinking of the cup product as an ``internal'' product: given
cohomology classes $\omega$ and $\eta$ on a space, we produce a new
class $\omega \smile \eta$.

But there is also an ``external'' product, i.e., given classes $\omega
\in H^\star(X)$ and $\eta \in H^\star(Y)$, we can produce a class in
$H^\star(X \times Y)$.  This then raises the question of how
$H^\star(X)$ and $H^\star(Y)$ are related to $H^\star(X \times Y)$.
That relationship is uncovered in the K\"unneth Theorem (or, as it is
often called, K\"unneth formula).  To dig into this, read the section
on the K\"unneth formula in \textsection 3.2 Cup Product from
Hatcher's \textit{Algebraic Topology}.  Then take a look at
\textsection 3.B at the end of Chapter~3.

\end{document}
