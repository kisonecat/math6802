\documentclass{homework}
\course{Math 6802}
\author{Jim Fowler}
\input{preamble}

\begin{document}
\maketitle

\begin{inspiration} % page 212, Notices of the AMS, Volume 56, Number 2
Some mathematicians are birds, others are frogs\ldots Mathematics needs both birds and frogs.
\byline{Freeman Dyson}
\end{inspiration}

\section{Terminology}

\begin{problem}
  What is the suspension $\Sigma X$ of a space X?
\end{problem}

\begin{problem}
  What is the \textbf{Hilbert--Poincar\'e series} of a graded vector space?  
\end{problem}

\section{Numericals}

\begin{problem}
Compute the Hilbert--Poincar\'e series of $H^\star(\CP^n \times \CP^m ; \mathbb{Q})$.
\end{problem}

\section{Exploration}

\begin{problem} Let $\alpha$ denote a generator of $H^1(S^1)$.  For a
finite CW complex $X$ and a class $\omega \in H^1(X)$, show that there
is a map $f : X \to S^1$ so that $\omega = f^\star(\alpha)$.
\end{problem}

\begin{problem}
  Conclude that if $\omega \in H^1(X)$, then $\omega \smile \omega = 0$.
  
  This is an example of reasoning by the ``univeral example'' of $S^1$.
\end{problem}

\begin{problem}
  Use the short exact sequence
  \[
    0 \to \Z/2\Z \to \Z/4\Z \to \Z/2\Z \to 0
  \]
  to produce a map $\beta : H^\star(X;\Z/2) \to H^{\star + 1}(X ; \Z/2)$.  This is a \textbf{Bockstein homomorphism}.
\end{problem}

\section{Prove or Disprove and Salvage if Possible (PODASIP)}

\begin{problem}
  The Bockstein $\beta : H^\star(\RP^\infty ; \Z/2\Z) \to H^{\star + 1}(\RP^\infty ; \Z/2\Z)$ vanishes.
\end{problem}

\begin{problem} % only for positive-degree classes
If $\omega, \eta \in H^\star(\Sigma X)$ then $\omega \smile \eta = 0$.

\textit{Hint:} consider \textit{relative} cup products.
\end{problem}

\end{document}
