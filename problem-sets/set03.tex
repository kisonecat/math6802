\documentclass{homework}
\course{Math 6802}
\author{Jim Fowler}
\input{preamble}

\begin{document}
\maketitle

\begin{inspiration}
  If I could only understand the beautiful consequence \\ following from the concise proposition $d^2 = 0$.
  \byline{Henri Cartan}
\end{inspiration}

\section{Terminology}

\begin{problem}
  For $\varphi \in H^k(X)$ and $\psi \in H^\ell(X)$, define the cup product $\varphi \cupp \psi \in H^{k + \ell}(X)$.
\end{problem}

\section{Numericals}

\begin{problem}
 For the orientable genus $g$ surface $\Sigma_g$, compute $H^\star(\Sigma_g)$ as a ring.
\end{problem}

\begin{problem}
 Compute $H^\star(S^2 \wedge S^4)$ as a ring.
\end{problem}

\begin{problem} Define $\epsilon_n := (-1)^{n \cdot (n+1)/2}$ and
verify $\epsilon_{i+j} = (-1)^{ij} \epsilon_i \epsilon_j$.
\end{problem}

% \begin{problem}
%   Consider $X \times I$ as a simplicial complex; in particular, for
%   $\Delta^n \times I$,
%   \begin{align*}
%     [v_0,\ldots,v_n] &= \Delta^n \times \{0\} \\
%     [w_0,\ldots,w_n] &= \Delta^n \times \{1\} \\
%     \Delta^n \times I &= \bigcup [v_0,\ldots,v_i,w_n,\ldots,w_i].
%     \end{align*}
%     Draw a picture to illustrate the resulting simplicial structure on the prism $\Delta^2 \times I$.
% \end{problem}

\section{Exploration}

\begin{problem}
Let $\tau : C_n(X) \to C_n(X)$ be the chain map which, on a simplex $\sigma$, is defined as
\[
\tau(\sigma) = \epsilon_n (\sigma \circ \mathrm{reverse}),
\]
where $\mathrm{reverse}$ is the affine map $[v_0,\ldots,v_n] \to
[v_n,\ldots,v_0]$.

Is $\tau$ a chain map?
\end{problem}

\begin{problem}
  Verify $\tau^\star \alpha \cupp \tau^\star \beta = \pm
  \,\tau^\star(\beta \cupp \alpha)$.

  Determine the sign.
\end{problem}


\begin{problem}
 Recall the \textbf{prism
    operator} $P : C_n(X) \to C_{n+1}(X)$ defined via
  \[
  P(\sigma) = \sum_i (-1)^i \epsilon_{n-i} (\sigma \circ \projection)
  |_{[v_0,\ldots,v_i,w_n,\ldots, w_i]},
  \]
  where $\mathrm{proj}$ is the projection $\Delta^n \times I \to \Delta^n$.

  Show that $\partial \circ P + P \circ \partial = \tau - \mathrm{Id}$.
\end{problem}

\begin{problem} Describe a four-fold covering map $f : \Sigma_5 \to
  \Sigma_2$.

  Then describe $f^\star : H^\star(\Sigma_2) \to H^\star(\Sigma_5)$ as a ring homomorphism.
\end{problem}

\section{Prove or Disprove and Salvage if Possible (PODASIP)}

\begin{problem}
  For $\varphi \in H^k(X)$ and $\psi \in H^\ell(X)$, we have \(\varphi \cupp \psi = \psi \cupp \varphi\).
\end{problem}

\begin{problem} If $f : \Sigma_i \to \Sigma_j$ and $g :
\Sigma_j \to \Sigma_i$ and $g \circ f = \id_{\Sigma_i}$ then $i =
j$.
\end{problem}


\end{document}




