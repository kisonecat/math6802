\documentclass{homework}
\course{Math 6802}
\author{Jim Fowler}
\input{preamble}

This problem set is \textit{quite different} from the rest and is
something of an experiment.  You've spent a lot of time this term
learning algebraic topology; can you use what you've learned to make
it easier for others to learn some algebraic topology?

Work in a small group to go go out into the real world, find some
incompletely or poorly explained algebraic topology, and fix that.
For example, add some text to Wikipedia improving its coverage of
algebraic topology.  Here are some suggestions of things you could do.
\begin{itemize}
\item Add missing citations to \url{https://en.wikipedia.org/wiki/Products_in_algebraic_topology}.
\item Add text about the fundamental class to \url{https://en.wikipedia.org/wiki/Cap_product}.
  \item Share solution sets for \url{https://github.com/kisonecat/math6802} by making a pull request with some nicely written solutions.
   \item Some subsections of \url{https://en.wikipedia.org/wiki/Fundamental_class} could be expanded.
\item The talk page for \url{https://en.wikipedia.org/wiki/Singular_homology} mentions a number of improvements that could be made to the singular homology page, such as
  \begin{itemize}
  \item adding a table of $H_\star$ for common spaces,
  \item adding a section on reduced homology, and
    \item mentioning the universal coefficient theorem.
    \end{itemize}
  \item Create a new article about ``collars'' in topology with a reference to Brown's paper.
  \item There's already an article about chains; create one on cochains.
  \item Add more examples to \url{https://en.wikipedia.org/wiki/Cohomology_ring}.
  \item Add real projective space as an example to \url{https://en.wikipedia.org/wiki/Cellular_homology}.
   \item Include figures in \url{https://en.wikipedia.org/wiki/Delta_set} to demonstrate how delta sets differ from simplicial complexes.
   \item Add additional applications to \url{https://en.wikipedia.org/wiki/Excision_theorem}.
     \item Applications of the \url{https://en.wikipedia.org/wiki/Nine_lemma} could be added.
\end{itemize}       
Of course, the best ideas are those you'll come up with.  Perhaps you were looking for something online and frustrated.

You can make these improvements anonymously, but to earn credit,
please email \texttt{fowler@math.osu.edu} privately describing your
contribution so I can record points in the gradebook.  You'll earn
full credit for even a small contribution.  In terms of expectations,
I encourage you to spend at least an hour on this project, and I
encourage you to work in groups.  It helps to have someone proofread
what you've written.

Given that this is a vague assignment, I understand you may have
questions so please email \texttt{fowler@math.osu.edu} with your
concerns.  I hope you find this an authentic task.

\end{document}
