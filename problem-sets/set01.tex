\documentclass{homework}
\course{Math 6802}
\author{Jim Fowler}
\input{preamble}

\begin{document}
\maketitle

\begin{inspiration}
``Homology was noticed by Aristotle (c. 350 BC), and was explicitly analysed by Pierre Belon in his 1555 \textit{Book of Birds}\ldots''
  \byline{The Wikipedia article for ``Homology''}
\end{inspiration}

\section{Terminology}

\begin{problem}
  Define $H_n(X,A;G)$.
\end{problem}

\begin{problem}
 Define  
  $H^n(X;G)$.
\end{problem}

\begin{problem}
  For a homeomorphism $f : X \to X$, define the mapping torus $M_f$.
\end{problem}

\section{Numericals}

\begin{problem} Let $T^2 := S^1 \times S^1$.  Then for a matrix $A \in
\GL_2(\Z)$, define a map $f_A : T^2 \to T^2$ and compute the homology
of $M_f$.
\end{problem}

\begin{problem}
  Compute $H_\star(\RP^2;\Z)$ and $H^\star(\RP^2;\Z)$.  Are they the same?
\end{problem}

\section{Exploration}

\begin{problem}
  Is it the case that $H_n(X;G) = H_n(X;\Z) \otimes G$?
\end{problem}

\begin{problem}
  Consider the simplicial graph $X$ with a vertex $a_i$ and $b_i$ for each $i \in \Z$, and edges
  \begin{itemize}
  \item between $a_k$ and $a_{k+1}$,
  \item between $b_k$ and $b_{k+1}$,
  \item between $a_k$ and $b_k$, for each $k \in \Z$.
  \end{itemize} Compute $H_\star(X;\Z)$ and $H^\star(X;\Z)$.  Are they
torsion-free?  Are they the same?
\end{problem}

\section{Prove or Disprove and Salvage if Possible (PODASIP)}

\textit{You may not have met these PODASIP-style problems before.
  What follows are statements which may be ``true'' or ``false.''  If
  the statement is true, you must provide a proof.  If the statement
  is false, explain why it is false, then fix the statement and
  finally supply a proof for the repaired statement.}

\begin{problem} If the CW complex $X$ only has cells in even
dimensions, then its homology groups $H_n(X)$ are torsion-free.
\end{problem}

\begin{problem} Suppose $0 \to A \to B \to C \to 0$ is an exact
sequence of abelian groups.  Then for an abelian group $G$, the
sequence
  \[
    0 \to A \otimes G \to B \otimes G \to C \otimes G \to 0
    \] is exact.
\end{problem}

\end{document}
